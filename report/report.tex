\include{settings}

\begin{document}	% начало документа

\begin{titlepage}	% начало титульной страницы

	\begin{center}		% выравнивание по центру

		\large Санкт-Петербургский Политехнический Университет Петра Великого\\
		\large Институт компьютерных наук и технологий \\
		\large Кафедра компьютерных систем и программных технологий\\[6cm]
		% название института, затем отступ 6см
		
		\huge Арифметические и логические основы вычислительной техники\\[0.5cm] % название работы, затем отступ 0,5см
		\large Приложение, которое переводит число, записанное в одной системе счисления, в другую систему счисления\\[11cm]

	\end{center}


	\begin{flushright} % выравнивание по правому краю
		\begin{minipage}{0.25\textwidth} % врезка в половину ширины текста
			\begin{flushleft} % выровнять её содержимое по левому краю

				\large\textbf{Работу выполнил:}\\
				\large Ламтев А.Ю.\\
				\large {Группа:} 23501/4\\

			\end{flushleft}
		\end{minipage}
	\end{flushright}
	
	\vfill % заполнить всё доступное ниже пространство

	\begin{center}
	\large Санкт-Петербург\\
	\large \the\year % вывести дату
	\end{center} % закончить выравнивание по центру

\thispagestyle{empty} % не нумеровать страницу
\end{titlepage} % конец титульной страницы

\vfill % заполнить всё доступное ниже пространство



\hypertarget{toc}
\tableofcontents
\newpage

\section*{Описание}
\addcontentsline{toc}{section}{Описание}

Приложение представляет из себя консольную утилиту, которая в виде аргументов командной строки получает входные данные и выводит результат на экран. 

Использование приложения:\\[0.1cm]

\textbf{notation\_translator} \textbf{[\textit{команда}]}\\[0.1cm]

\textbf{Команды:}
\begin{itemize}

\item \textbf{-h},\textbf{-help},\textbf{--help}

	На экран будет выведена информация помощника

\item \textbf{-v},\textbf{-version},\textbf{--version}

	На экран будет выведена версия приложения
	
\item \textbf{<дробное число> <начальная сс> <конечная сс> [\textit{точность}]}

	Где
	\begin{itemize}
	\item \textbf{<дробное число>}
	
		Дробное число, которое нужно перевести.
		
		Записывается в виде \textbf{<целая часть,дробная часть>}.
		
		Ограничения по размеру нет.
		
	\item \textbf{<начальная сс>}
	
		Система счисления, в которой изначально представлено число.
		
		Принимаемые значения: 2 - 36.
		
	\item \textbf{<конечная сс>}
		
		Система счисления, в которую необходимо перевести число.
		
		Принимаемые значения: 2 - 36.
	
	\item[\textbf{--}] \textbf{[\textit{точность}]}
	
		Число знаков после запятой в результирующем числе.
		
		Не является обязательным параметром.
		
		Значение по умолчанию: 8.
		
		Принимаемые значения: любые.
		
	\end{itemize}
	 
\end{itemize}

\section*{Листинги}
\addcontentsline{toc}{section}{Листинги}

\subsection*{Ядро}
\addcontentsline{toc}{subsection}{Ядро}

\subsubsection*{Бизнес-логика}
\addcontentsline{toc}{subsubsection}{Бизнес-логика}

\captionof{lstlisting}{StringParser.java}
\lstinputlisting[label=code:hello]{../core/src/main/java/com/lamtev/notation_translator/core/StringParser.java}


\captionof{lstlisting}{StringValidator.java}
\lstinputlisting[label=code:hello]{../core/src/main/java/com/lamtev/notation_translator/core/StringValidator.java}


\captionof{lstlisting}{Translator.java}
\lstinputlisting[label=code:hello]{../core/src/main/java/com/lamtev/notation_translator/core/Translator.java}


\captionof{lstlisting}{StringCompiler.java}
\lstinputlisting[label=code:hello]{../core/src/main/java/com/lamtev/notation_translator/core/StringCompiler.java}

\subsubsection*{Модульные тесты}
\addcontentsline{toc}{subsubsection}{Модульные тесты}

\captionof{lstlisting}{StringParserTest.java}
\lstinputlisting[label=code:hello]{../core/src/test/java/com/lamtev/notation_translator/core/StringParserTest.java}


\captionof{lstlisting}{StringValidatorTest.java}
\lstinputlisting[label=code:hello]{../core/src/test/java/com/lamtev/notation_translator/core/StringValidatorTest.java}


\captionof{lstlisting}{TranslatorTest.java}
\lstinputlisting[label=code:hello]{../core/src/test/java/com/lamtev/notation_translator/core/TranslatorTest.java}


\captionof{lstlisting}{StringCompilerTest.java}
\lstinputlisting[label=code:hello]{../core/src/test/java/com/lamtev/notation_translator/core/StringCompilerTest.java}


\subsection*{Консольное приложение}
\addcontentsline{toc}{subsection}{Консольное приложение}

\captionof{lstlisting}{Utility.java}
\lstinputlisting[label=code:hello]{../cli/src/main/java/com/lamtev/notation_translator/cli/Utility.java}


\captionof{lstlisting}{Application.java}
\lstinputlisting[label=code:hello]{../cli/src/main/java/com/lamtev/notation_translator/cli/Application.java}


\captionof{lstlisting}{Helper.java}
\lstinputlisting[label=code:hello]{../cli/src/main/java/com/lamtev/notation_translator/cli/Helper.java}

\end{document}

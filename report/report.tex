\include{settings}

\begin{document}	% начало документа

\begin{titlepage}	% начало титульной страницы

	\begin{center}		% выравнивание по центру

		\large Санкт-Петербургский Политехнический Университет Петра Великого\\
		\large Институт компьютерных наук и технологий \\
		\large Кафедра компьютерных систем и программных технологий\\[6cm]
		% название института, затем отступ 6см
		
		\huge Арифметические и логические основы вычислительной техники\\[0.5cm] % название работы, затем отступ 0,5см
		\large Приложение, которое переводит число, записанное в одной системе счисления, в другую систему счисления\\[11cm]

	\end{center}


	\begin{flushright} % выравнивание по правому краю
		\begin{minipage}{0.25\textwidth} % врезка в половину ширины текста
			\begin{flushleft} % выровнять её содержимое по левому краю

				\large\textbf{Работу выполнил:}\\
				\large Ламтев А.Ю.\\
				\large {Группа:} 23501/4\\

			\end{flushleft}
		\end{minipage}
	\end{flushright}
	
	\vfill % заполнить всё доступное ниже пространство

	\begin{center}
	\large Санкт-Петербург\\
	\large \the\year % вывести дату
	\end{center} % закончить выравнивание по центру

\thispagestyle{empty} % не нумеровать страницу
\end{titlepage} % конец титульной страницы

\vfill % заполнить всё доступное ниже пространство



% Содержание
\tableofcontents
\newpage



\section{Цель работы}


\section{Программа работы}


\section{Теоретическая информация}


\section{Ход выполнения работы}

\subsection{Список}

\begin{itemize}
\item первый элемент списка
\item второй элемент списка
\end{itemize}


%\subsection{Картинка}

%\begin{figure}[H]
%	\begin{center}
%		\includegraphics[scale=0.7]{pics/sample}
%		\caption{название картинки} 
%		\label{pic:pic_name} % название для ссылок внутри кода
%	\end{center}
%\end{figure}


\subsection{Листинг}

\captionof{lstlisting}{Utility.java} % для печати символ '_' требует выходной символ '\'
\lstinputlisting[label=code:hello]{../cli/src/main/java/com/lamtev/notation_translator/cli/Utility.java}
\parindent=1cm % командна \lstinputlisting сбивает параментры отступа
Текст без отступа (следует за вставкой)

Новый параграф

\noindent Новый параграф с принудительно выключенным отступом


\subsection{Частичный листинг}
% настрока частичного ввода (требуется один раз)
\makeatletter
\def\lst@PlaceNumber{\llap{\normalfont
                \lst@numberstyle{\the\lst@lineno}\kern\lst@numbersep}}
\makeatother

\captionof{lstlisting}{фрагмент Utility.java}
\lstinputlisting[label=code:hello_mod, linerange={6-7}]{../cli/src/main/java/com/lamtev/notation_translator/cli/Utility.java}
\parindent=1cm

\subsection{Таблица}

\begin{table}[H]
	\begin{center}
		\begin{tabular}{|l|l|}
			\hline
			top left & top right\\ \hline
			bot left & bot right\\ \hline
		\end{tabular}
		\caption{ Название таблицы}
		\label{tabular:tab_examp}
	\end{center}
\end{table}

\section{Выводы}
 Всё классно
\end{document}
